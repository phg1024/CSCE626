\documentclass{article}
\usepackage{amsmath,amssymb, amsthm,latexsym,paralist}

\usepackage{graphicx}

\usepackage{listings}
\usepackage{color}

\definecolor{dkgreen}{rgb}{0,0.6,0}
\definecolor{gray}{rgb}{0.5,0.5,0.5}
\definecolor{mauve}{rgb}{0.58,0,0.82}

\lstset{frame=tb,
  language=c++,
  aboveskip=3mm,
  belowskip=3mm,
  showstringspaces=false,
  columns=flexible,
  basicstyle={\small\ttfamily},
  numbers=none,
  numberstyle=\tiny\color{gray},
  keywordstyle=\color{blue},
  commentstyle=\color{dkgreen},
  stringstyle=\color{mauve},
  breaklines=true,
  breakatwhitespace=true
  tabsize=3
}

\theoremstyle{definition}
\newenvironment{problem}[1]{\noindent\textbf{Problem #1.}}{\\}
\newenvironment{solution}{\noindent\textbf{Solution.}}{}
\newenvironment{solutionitem}[1]{#1}{\bigbreak}

\newcommand{\name}[1]{\noindent\textbf{Name}: #1}
\newcommand{\uin}[1]{\noindent\textbf{UIN}: #1\\}

\newcommand{\problemset}[1]{\begin{center}\textbf{Homework #1}\end{center}}
\newcommand{\duedate}[2]{\begin{quote}\textbf{Due dates:} Electronic
    submission of hw1.tex and hw1.pdf files of this homework is due on
    \textbf{#1} on \texttt{https://csnet.cs.tamu.edu}.  A signed paper copy of the pdf file is due on
    \textbf{#2} at the beginning of class.\end{quote} }


\begin{document}
\begin{center}
{\large
CSCE 626 - Parallel Algorithm Design and Analysis}
\end{center}
\problemset{1}
\name{ Peihong Guo }
\uin{421003404}
\begin{problem}{1}  Given a sequence of numbers $x_1, x_2, \ldots, x_n$, the prefix sums are the partial sums
\[
\begin{split}
& s_1 = x_1 \\
& s_2 = x_1 + x_2 \\
& \ldots \\
& s_n = x_1 + x_2 + ... + x_n
\end{split}
\]
Describe an algorithm to compute the prefix sums on a PRAM with $n$
processors in $O(\log n)$ time. Analyze the running time of your
algorithm and argue, at least informally, its correctness.
Which PRAM model does your algorithm use (e.g., EREW, CREW, CRCW)?
Does your algorithm require a synchronous PRAM?
\end{problem}

\begin{solution}
\begin{solutionitem}{(a)}
\end{solutionitem}
\end{solution}

\begin{problem}{2} Repeat the previous question, but this time consider
  the case when $p < n$, i.e., the number of processors is less than
  the number of input elements. You should design the best algorithm
  you can in terms of time and work. Analyze the running time of your
  algorithms and argue, at least informally, their correctness.
  (If an algorithm is similar to the corresponding one from 2, you need only discuss the modifications.)
\end{problem}


\goodbreak
%\checklist
\end{document}
